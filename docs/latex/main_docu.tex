\documentclass[fleqn]{article}

\usepackage[a4paper,margin=1cm]{geometry}
\usepackage[utf8]{inputenc}
\usepackage{amsmath,xfrac}
\usepackage[dvipsnames]{xcolor}
\usepackage{underscore}


% ICON4Py - ICON inspired code in Python and GT4Py
%
% Copyright (c) 2022-2024, ETH Zurich and MeteoSwiss
% All rights reserved.
%
% Please, refer to the LICENSE file in the root directory.
% SPDX-License-Identifier: BSD-3-Clause

%===============================================================================
% This file contains macro definitions that can be used in docstrings and are
% parsed by the MathJax processor within Sphinx.
% These macros are loaded into `mathjax3_config` in conf.py

% The predefined color names are defined on this page
% https://en.wikibooks.org/wiki/LaTeX/Colors
% both the standard as well as the dvipsnames can be used

\newcommand{\ibm}{immersed boundary method}
\newcommand{\IBM}{IBM}
\newcommand{\ie}{i.e.,\ }

%===============================================================================
% Aesthetic macros

% Glyph sizes
\newcommand{\superscriptsize}{\scriptscriptstyle}
\newcommand{\subscriptsize}{\scriptscriptstyle}

% Colors
\newcommand{\timecolor}[1]{\color{teal}{#1}}
\newcommand{\vertcolor}[1]{\color{YellowOrange}{#1}}
\newcommand{\horicolor}[1]{\color{RedViolet}{#1}}

%\renewcommand{\sum}{{\horicolor{\Large{\Sigma}}}} % move sub/superscripts of sum symbol
\renewcommand{\sum}{{\horicolor{\text{\Large{$\Sigma$}}}}} % move sub/superscripts of sum symbol

%===============================================================================
% Numerics

% Discretization points
\newcommand{\timestep}[1]{{\subscriptsize{\timecolor{#1}}}}
\newcommand{\vertlevel}[1]{{\subscriptsize{\vertcolor{#1}}}}
\newcommand{\horipoint}[1]{{\subscriptsize{\horicolor{#1}}}}

% Vectors
\newcommand{\vect}[1]{\overrightarrow{#1}}
\newcommand{\vers}[1]{\hat{#1}}

% Differentials and deltas
\newcommand{\dif}{\mathrm{d}}
\newcommand{\Dt}{{\Delta t}}
\newcommand{\Dz}[1]{{\Delta z}_\vertlevel{#1}}
\newcommand{\divhori}[1]{\vec{\nabla}\!_h\cdot\left(#1\right)}
%\newcommand{\Gradn}{{\horicolor{\Large{\Delta}}}}
\newcommand{\Gradn}{{\horicolor{\text{\Large{$\Delta$}}}}}

% Derivatives
\newcommand{\ddt}[1]{\frac{\dif #1}{\dif t}}
\newcommand{\ddx}[1]{\frac{\dif #1}{\dif x}}
\newcommand{\ddy}[1]{\frac{\dif #1}{\dif y}}
\newcommand{\ddz}[1]{\frac{\dif #1}{\dif z}}
\newcommand{\ddxn}[1]{\frac{\dif #1}{\dif n}}
\newcommand{\ddxt}[1]{\frac{\dif #1}{\dif t}}
\newcommand{\pdt}[1]{\frac{\partial #1}{\partial t}}
\newcommand{\pdx}[1]{\frac{\partial #1}{\partial x}}
\newcommand{\pdy}[1]{\frac{\partial #1}{\partial y}}
\newcommand{\pdz}[1]{\frac{\partial #1}{\partial z}}
\newcommand{\pdzz}[1]{\frac{\partial^2 #1}{\partial z^2}}
\newcommand{\pdxn}[1]{\frac{\partial #1}{\partial n}}
\newcommand{\pdxt}[1]{\frac{\partial #1}{\partial t}}
\newcommand{\lapl}[1]{\nabla^2\!\left(#1\right)}
\newcommand{\lapllapl}[1]{\nabla^4\!\left(#1\right)}

% Variables with helper macros for formatting superscript and subscript
%
% temporal, horizontal and vertical
\newcommand{\variable}[4]{{{#1}_{ \horipoint{#3}\,\vertlevel{#4}}^{\timestep{#2}}}}
\newcommand{\temparray}[3]{       \variable{t_a                                  }{#1}{#2}{#3}}
\newcommand{\windu}[3]{           \variable{u                                    }{#1}{#2}{#3}}
\newcommand{\windv}[3]{           \variable{v                                    }{#1}{#2}{#3}}
\newcommand{\vn}[3]{              \variable{v_n                                  }{#1}{#2}{#3}}
\newcommand{\vnlapl}[3]{          \variable{\lapl{\vn{#1}{}{}}                   }{  }{#2}{#3}}
\newcommand{\vnlapllapl}[3]{      \variable{\lapllapl{\vn{#1}{}{}}               }{  }{#2}{#3}}
\newcommand{\vt}[3]{              \variable{v_t                                  }{#1}{#2}{#3}}
\newcommand{\vh}[3]{              \variable{\vec{v_h}                            }{#1}{#2}{#3}}
\newcommand{\w}[3]{               \variable{w                                    }{#1}{#2}{#3}}
\newcommand{\wlapl}[3]{           \variable{\lapl{\w{#1}{}{}}                    }{  }{#2}{#3}}
\newcommand{\wcc}[3]{             \variable{w_{\text{cc}}                        }{#1}{#2}{#3}}
\newcommand{\kinehori}[3]{        \variable{K_h                                  }{#1}{#2}{#3}}
\newcommand{\advvn}[3]{           \variable{\text{adv}(v_n)                      }{#1}{#2}{#3}}
\newcommand{\vortvert}[3]{        \variable{\zeta                                }{#1}{#2}{#3}}
\newcommand{\density}[3]{         \variable{\rho                                 }{#1}{#2}{#3}}
\newcommand{\densityprime}[3]{    \variable{\rho'                                }{#1}{#2}{#3}}
\newcommand{\densityprimedx}[3]{  \variable{\pdx{\densityprime{#1}{}{}}          }{  }{#2}{#3}}
\newcommand{\densityprimedy}[3]{  \variable{\pdy{\densityprime{#1}{}{}}          }{  }{#2}{#3}}
\newcommand{\pres}[3]{            \variable{P                                    }{#1}{#2}{#3}}
\newcommand{\vtemp}[3]{           \variable{T_v                                  }{#1}{#2}{#3}}
\newcommand{\vtempprime}[3]{      \variable{T_v'                                 }{#1}{#2}{#3}}
\newcommand{\vpotemp}[3]{         \variable{\theta_v                             }{#1}{#2}{#3}}
\newcommand{\vpotempprime}[3]{    \variable{\theta_v'                            }{#1}{#2}{#3}}
\newcommand{\vpotempprimedx}[3]{  \variable{\pdx{\vpotempprime{#1}{}{}}          }{  }{#2}{#3}}
\newcommand{\vpotempprimedy}[3]{  \variable{\pdy{\vpotempprime{#1}{}{}}          }{  }{#2}{#3}}
\newcommand{\vpotempprimelapl}[3]{\variable{\lapl{\vpotempprime{#1}{}{}}         }{  }{#2}{#3}}
\newcommand{\exner}[3]{           \variable{\pi                                  }{#1}{#2}{#3}}
\newcommand{\exnerprime}[3]{      \variable{\pi'                                 }{#1}{#2}{#3}}
\newcommand{\exnerprimegradh}[3]{ \variable{\pdxn{\exnerprime{#1}{}{}}           }{  }{#2}{#3}}
\newcommand{\exnerprimedz}[3]{    \variable{\pdz{\exnerprime{#1}{}{}}            }{  }{#2}{#3}}
\newcommand{\exnerprimedzz}[3]{   \variable{\frac{1}{2}\pdzz{\exnerprime{#1}{}{}}}{  }{#2}{#3}}
\newcommand{\Khsmag}[3]{          \variable{{K_h^\text{smag}}                    }{#1}{#2}{#3}}
%
% horizontal and vertical
\newcommand{\variableXK}[3]{{{#1}_{\horipoint{#2}\,\vertlevel{#3}}}}
\newcommand{\densityref}[2]{\variableXK{\rho                                                           }{#1}{#2}}
\newcommand{\presref}[2]{   \variableXK{P_0                                                            }{#1}{#2}}
\newcommand{\vpotempref}[2]{\variableXK{\theta_{v0}                                                    }{#1}{#2}}
\newcommand{\exnerref}[2]{  \variableXK{\pi_0                                                          }{#1}{#2}}
\newcommand{\dexrefdz}[2]{  \variableXK{\frac{1}{\Dz{}\vpotempref{}{}}\ddz{\exnerref{}{}}              }{#1}{#2}}
\newcommand{\ddexrefdzz}[2]{\variableXK{\ddz{}\left(\frac{1}{\vpotempref{}{}}\ddz{\exnerref{}{}}\right)}{#1}{#2}}
%
% horizontal
\newcommand{\variableX}[2]{{{#1}_{\horipoint{#2}}}}
\newcommand{\coriolis}[1]{       \variableX{f                                                                 }{#1}}
\newcommand{\exnhydrocorr}[1]{   \variableX{\frac{g}{\cpd \vpotemp{\color{black}{2}}{}{}}\pdxn{\vpotemp{}{}{}}}{#1}}
\newcommand{\orientationnorm}[1]{\variableX{\vers{n}}{#1}}
\newcommand{\orientationnormx}[1]{\variableX{\vers{n_x}}{#1}}
\newcommand{\orientationnormy}[1]{\variableX{\vers{n_y}}{#1}}
\newcommand{\orientationtang}[1]{\variableX{\vers{t}}{#1}}
\newcommand{\orientationtangx}[1]{\variableX{\vers{t_x}}{#1}}
\newcommand{\orientationtangy}[1]{\variableX{\vers{t_y}}{#1}}

% Physical constants
\newcommand{\Rd}{R_{d}}
\newcommand{\cpd}{c_{pd}}
\newcommand{\cvd}{c_{vd}}

% Numerical constants
\newcommand{\nlev}{\mathtt{num\_lev}}
\newcommand{\nflatlev}{\mathtt{flat\_lev}}
\newcommand{\nflatgradp}{\mathtt{flat\_gradp\_lev}}
\newcommand{\smaglimit}{\mathtt{smag\_limit}}
\newcommand{\smagoffset}{\mathtt{smag\_offset}}

% Temporal steps
\newcommand{\n}{{\timecolor{\text{n}}}}
\newcommand{\ntilde}{{\timecolor{\tilde{\n}}}}

% Vertical levels
\renewcommand{\k}{{\vertcolor{\text{k}}}}

% Horizontal discretization
\renewcommand{\c}{{\horicolor{\text{c}}}}
\newcommand{\e}{{\horicolor{\text{e}}}}
\renewcommand{\v}{{\horicolor{\text{v}}}}

% Horizontal offsets
\newcommand{\offProv}[1]{\horicolor{#1}}

% Inter/extrapolation and derivative weights
\newcommand{\cellarea}{a_c}
\newcommand{\edgearea}{a_e}
\newcommand{\Wrbf}{\overset{\scriptscriptstyle{\text{rbf}}}{\chi}}
\newcommand{\Wlev}{\overset{\scriptscriptstyle{\text{lev}}}{\chi}}
\newcommand{\Whor}{\overset{\scriptscriptstyle{\text{hor}}}{\chi}}
\newcommand{\Wedge}{\frac{1}{d_{12}}}
\newcommand{\Cgrad}{\overset{\scriptscriptstyle{\text{grad}}}{\chi}}
\newcommand{\Crot}{\overset{\scriptscriptstyle{\text{rot}}}{\chi}}
\newcommand{\Cdiv}{\overset{\scriptscriptstyle{\text{div}}}{\chi}}
\newcommand{\dzgradp}{(h^* - h_k)}
\newcommand{\WtimeExner}{\gamma}
\newcommand{\WtimeVWind}{\eta}
\newcommand{\Ledge}{\ell_{\text{e}}}
\newcommand{\Lvertvert}{\ell_{\text{vv}}}

% Indexes
\newcommand{\IDXpg}{\overset{\scriptstyle{\partial \pi / \partial n}}{\text{IDX}}}

% Code variables
\newcommand{\Zdensityexpl}{Z^{\density{}{}{}\text{expl}}}
\newcommand{\Zexnerexpl}{Z^{\exner{}{}{}\text{expl}}}
\newcommand{\zflxdivmass}{\mathtt{z\_flxdiv\_mass}}
\newcommand{\zflxdivtheta}{\mathtt{z\_flxdiv\_theta}}
\newcommand{\zcontrwfll}{\mathtt{z\_contr\_w\_fl\_l}}
\newcommand{\massfle}{\mathtt{mass\_fl\_e}}
\newcommand{\zthetave}{\mathtt{z\_theta\_v\_e}}
\newcommand{\zthetavfle}{\mathtt{z\_theta\_v\_fl\_e}}
\newcommand{\zvnavg}{\mathtt{z\_vn\_avg}}
\newcommand{\ddtexnerphy}{\mathtt{ddt\_exner\_phy}}
\newcommand{\geofacgrgx}{\mathtt{geofac\_grg\_x}}
\newcommand{\geofacgrgy}{\mathtt{geofac\_grg\_y}}
%
\newcommand{\diffmultfacsmag}{\mathtt{diff\_multfac\_smag}}
\newcommand{\diffmultfacvn}{\mathtt{diff\_multfac\_vn}}
\newcommand{\diffmultfacw}{\mathtt{diff\_multfac\_w}}
\newcommand{\khsmagone}{\mathtt{kh\_smag\_1}}
\newcommand{\khsmagtwo}{\mathtt{kh\_smag\_2}}
\newcommand{\dvtnorm}{\mathtt{dvt\_norm}}
\newcommand{\dvttang}{\mathtt{dvt\_tang}}
\newcommand{\nabvtang}{\mathtt{nabv\_tang}}
\newcommand{\nabvnorm}{\mathtt{nabv\_norm}}


\setlength{\parindent}{0pt}


\title{ICON4Py and IBM documentation}
\author{Jacopo Canton}
\date{\today}


%%%%%%%%%%%%%%%%%%%%%%%%%%%%%%%%%%%%%%%%%%%%%%%%%%%%%%%%%%%%%%%%%%%%%%%%%%%%%%%%%%%%%%%%%%%%%%%%%%%%%%%%%%%%%%%%%%%%%%%%
\begin{document}

\maketitle

\begin{abstract}
Work in progress documentation of the immersed boundary method implementation in ICON.
\end{abstract}

\section{Equations}

%=======================================================================================================================
\clearpage
\subsection{Normal wind}
\label{sub:normal_wind}
%=======================================================================================================================


%-------------------------------------------------------------------------------
\paragraph{Advection}\

wind and kinetic energy
\begin{align}
  &\vt{\n}{\e}{\k} = \sum_{\offProv{e2c2e}} \Wrbf \vn{\n}{\e}{\k} \\
  &\kinehori{\n}{\e}{\k} = \frac{1}{2} \left( \vn{\n}{\e}{\k}^2 + \vt{\n}{\e}{\k}^2 \right) \\
  &\kinehori{\n}{\c}{\k} = \sum_{\offProv{c2e}} \Whor \kinehori{\n}{\e}{\k} \\
  &\vn{\n}{\e}{\k-1/2} = \Wlev \vn{\n}{\e}{\k} + (1 - \Wlev) \vn{\n}{\e}{\k-1} \\
\end{align}

vorticity
\begin{equation}
  \vortvert{\n}{\v}{\k} = \sum_{\offProv{v2e}} \Crot \vn{\n}{\e}{\k}
\end{equation}

contravariant-corrected vertical wind
\begin{align}
  &\wcc{\n}{\e}{\k} = \vn{\n}{\e}{\k} \pdxn{z} + \vt{\n}{\e}{\k} \pdxt{z} \\
  &\wcc{\n}{\c}{\k} = \sum_{\offProv{c2e}} \Whor \wcc{\n}{\e}{\k} \\
  &\wcc{\n}{\c}{\k-1/2} = \Wlev \wcc{\n}{\c}{\k} + (1 - \Wlev) \wcc{\n}{\c}{\k-1} \\
  &(\w{\n}{\c}{\k-1/2} - \wcc{\n}{\c}{\k-1/2}) =
    \begin{cases}
      \w{\n}{\c}{\k-1/2},                        & \k \in [0, \nflatlev+1)     \\
      \w{\n}{\c}{\k-1/2} - \wcc{\n}{\c}{\k-1/2}, & \k \in [\nflatlev+1, \nlev) \\
      0,                                         & \k = \nlev
    \end{cases}\\
  &(\w{\n}{\c}{\k} - \wcc{\n}{\c}{\k}) = \frac{1}{2} \left[(\w{\n}{\c}{\k-1/2} - \wcc{\n}{\c}{\k-1/2})
                                                         + (\w{\n}{\c}{\k+1/2} - \wcc{\n}{\c}{\k+1/2})\right] \\
\end{align}

sum all contributions
\begin{equation}
  \begin{split}
      \advvn{\n}{\e}{\k} & = \pdxn{\kinehori{}{}{}} + \vt{}{}{} (\vortvert{}{}{} + \coriolis{}) + \pdz{\vn{}{}{}} (\w{}{}{} - \wcc{}{}{}) \\
                         & = \Gradn_{\offProv{e2c}} \Cgrad \kinehori{\n}{c}{\k} + \kinehori{\n}{\e}{\k} \Gradn_{\offProv{e2c}} \Cgrad     \\
                         & + \vt{\n}{\e}{\k} (\coriolis{\e} + 1/2 \sum_{\offProv{e2v}} \vortvert{\n}{\v}{\k})                             \\
                         & + \frac{\vn{\n}{\e}{\k-1/2} - \vn{\n}{\e}{\k+1/2}}{\Dz{k}}
      \sum_{\offProv{e2c}} \Whor (\w{\n}{\c}{\k} - \wcc{\n}{\c}{\k})
  \end{split}
\end{equation}

%-------------------------------------------------------------------------------
\paragraph{Exner}

\begin{align}
  &\exnerprime{\ntilde}{\c}{\k} = (1 + \WtimeExner) \exnerprime{\n}{\c}{\k} - \WtimeExner \exnerprime{\n-1}{\c}{\k} \\
  &\exnerprime{\ntilde}{\c}{\k-1/2} = \Wlev \exnerprime{\ntilde}{\c}{\k} + (1 - \Wlev) \exnerprime{\ntilde}{\c}{\k-1}, \quad \k \in [\max(1,\nflatlev), \nlev) \\
  &\exnerprime{\ntilde}{\c}{\nlev-1/2} = \sum_{\k=\nlev-1}^{\nlev-3} \Wlev_{\k} \exnerprime{\ntilde}{\c}{\k}
\end{align}

horizontal gradient (at constant height) of $\exnerprime{}{}{}$ on flat and non-flat levels
\begin{align}
  &\exnerprimegradh{\ntilde}{\e}{\k} = \Cgrad \Gradn_{\offProv{e2c}} \exnerprime{\ntilde}{\c}{\k}, \quad \k \in [0, \nflatlev) \\
  &\begin{aligned}
    &\exnerprimegradh{\ntilde}{\e}{\k} &&= \left.\pdxn{\exnerprime{}{}{}}\right|_{s} - \left.\pdxn{h}\right|_{s}\exnerprimedz{}{}{}, \quad \k \in [\nflatlev, \nflatgradp]\\
    &                                  &&= \Wedge \Gradn_{\offProv{e2c}} \exnerprime{\ntilde}{\c}{\k}
                                         - \pdxn{h} \sum_{\offProv{e2c}} \Whor \exnerprimedz{\ntilde}{\c}{\k}
  \end{aligned} \\
  &\begin{aligned}
    &\exnerprimegradh{\ntilde}{\e}{\k} &&= \Wedge (\exnerprime{*}{\c_1}{} - \exnerprime{*}{\c_0}{}), \quad \k \in [\nflatgradp+1, \nlev) \\
    &                                  &&= \Wedge \Gradn_{\offProv{e2c}} \left[ \exnerprime{\ntilde}{\c}{\k^*} + \dzgradp \left( \exnerprimedz{\ntilde}{\c}{\k^*} + \dzgradp \exnerprimedzz{\ntilde}{\c}{\k^*} \right) \right]
  \end{aligned}
\end{align}

Hydrostatic correction term to $\exnerprimegradh{}{}{}$.
% This is only applied to edges for which the adjacent cell center (horizontally, not terrain-following) would be underground, i.e. edges in the $\IDXpg$ set.
% $\c_i$ are the indexes of the adjacent cell centers using $\offProv{e2c}$; $k^*$ is the level index of the neighboring (horizontally, not terrain-following) cell center and $h^*$ is its height.
\begin{align}
  &\vpotemp{}{\c_i}{\k} = \vpotemp{}{\c_i}{\k^*} + \dzgradp \frac{\vpotemp{}{\c_i}{\k^*-1/2} - \vpotemp{}{\c_i}{\k^*+1/2}}{\Dz{\k^*}} \\
  &\exnhydrocorr{\e} = \frac{g}{\cpd} \Wedge 4 \frac{ \vpotemp{}{\c_1}{\k} - \vpotemp{}{\c_0}{\k} }{ (\vpotemp{}{\c_1}{\k} + \vpotemp{}{\c_0}{\k})^2 } \\
  &\exnerprimegradh{\ntilde}{\e}{\k} = \exnerprimegradh{\ntilde}{\e}{\k} + \exnhydrocorr{\e} (h_k - h_{k^*}), \quad \e \in \IDXpg
\end{align}

%-------------------------------------------------------------------------------
\paragraph{Miura scheme}\

$\vpotemp{\n}{\e}{\k}$ is computed with the Miura scheme \S\ref{sub:miura}

%-------------------------------------------------------------------------------
\paragraph{Final update}

\begin{equation}
  \vn{\n+1^*}{\e}{\k} = \vn{\n}{\e}{\k} - \Dt \left( \advvn{\n}{\e}{\k} + \cpd \vpotemp{\n}{\e}{\k} \exnerprimegradh{\ntilde}{\e}{\k} \right)
\end{equation}

%=======================================================================================================================
\clearpage
\subsection{Vertical wind}
\label{sub:vertical_wind}
%=======================================================================================================================

%-------------------------------------------------------------------------------
\paragraph{Exner}

\begin{equation}
  \exnerprimedz{\ntilde}{\c}{\k} \approx \frac{\exnerprime{\ntilde}{\c}{\k-1/2} - \exnerprime{\ntilde}{\c}{\k+1/2}}{\Dz{\k}}
\end{equation}

%=======================================================================================================================
\clearpage
\subsection{Exner}
\label{sub:exner}
%=======================================================================================================================

%-------------------------------------------------------------------------------
\paragraph{Coefficients}

\begin{align}
  & \alpha_{\k??} = \WtimeVWind \density{\n}{\c}{\k-1/2} \vpotempprime{\n}{\c}{\k-1/2} \\
  & \beta_{\k??} = \Dt \frac{\Rd}{\cvd} \frac{\Gamma^\n_\k}{\Dz{}} = \Dt \frac{\Rd}{\cvd} \frac{\exnerprime{\n}{\c}{\k}}{\density{\n}{\c}{\k}\vpotempprime{\n}{\c}{\k}} \frac{1}{\Dz{}}
\end{align}

%-------------------------------------------------------------------------------
\paragraph{Average normal wind}\

``needed to obtain a nearly 2nd order accurate divergence (Zangl 2015)''

\begin{equation}
  \zvnavg_{\e\k} = \sum_{\offProv{e2c2eo}} \Whor \vn{\n+1^*}{\e}{\k}
\end{equation}

%-------------------------------------------------------------------------------
\paragraph{Miura scheme}\

$\density{\n}{\e}{\k}$ is computed with the Miura scheme \S\ref{sub:miura}

%-------------------------------------------------------------------------------
\paragraph{Explicit term}

\begin{align}
  & \zthetavfle_{\e\k} = \massfle_{\e\k} \zthetave_{\e\k} = \density{\n}{\e}{\k} \zvnavg_{\e\k} / \Dz{} \zthetave_{\e\k} \\
  & \zflxdivtheta_{\c\k} = \sum_{\offProv{c2e}} \Cdiv \zthetavfle_{\e\k}
\end{align}

\begin{equation}
  \begin{split}
    \Zexnerexpl = \exnerprime{\n}{\c}{\k} - \beta_{\k??}
      &\left( \zflxdivtheta_{\c\k} + \vpotempprime{\n}{\c}{\k-1/2} \zcontrwfll_{\k-1/2} \right.
      \\
      &\left.- \vpotempprime{\n}{\c}{\k+1/2} \zcontrwfll_{\k+1/2} \right) + \Dt \ddtexnerphy
  \end{split}
\end{equation}

%-------------------------------------------------------------------------------
\paragraph{Final update}

\begin{equation}
  \exnerprime{\n+1^*}{\c}{\k} = \Zexnerexpl + \exnerref{\c}{\k} - \beta \left( \alpha_{\k-1/2}\w{\n+1^*}{\c}{\k-1/2} -  \alpha_{\k+1/2}\w{\n+1^*}{\c}{\k+1/2} \right)
\end{equation}

%=======================================================================================================================
\clearpage
\subsection{Density}
\label{sub:density}
%=======================================================================================================================

%-------------------------------------------------------------------------------
\paragraph{Average normal wind}\

``needed to obtain a nearly 2nd order accurate divergence (Zangl 2015)''

\begin{equation}
  \zvnavg_{\e\k} = \sum_{\offProv{e2c2eo}} \Whor \vn{\n+1^*}{\e}{\k}
\end{equation}

%-------------------------------------------------------------------------------
\paragraph{Miura scheme}\

$\density{\n}{\e}{\k}$ is computed with the Miura scheme \S\ref{sub:miura}

%-------------------------------------------------------------------------------
\paragraph{Explicit term}

\begin{align}
  & \massfle_{\e\k} = \density{\n}{\e}{\k} \zvnavg_{\e\k} / \Dz{} \\
	& \zflxdivmass_{\c\k} = \sum_{\offProv{c2e}} \Cdiv \massfle_{\e\k} \\
  & \Zdensityexpl = \density{\n}{\c}{\k} - \frac{\Dt}{\Dz{}}\left( \zflxdivmass_{\c\k} + \zcontrwfll_{\c\k-1/2} - \zcontrwfll_{\c\k+1/2}\right)
\end{align}

%-------------------------------------------------------------------------------
\paragraph{Final update}

\begin{equation}
  \density{\n+1^*}{\c}{\k} = \Zdensityexpl - \WtimeVWind \frac{\Dt}{\Dz{}} \left(\density{\n}{\c}{\k-1/2} \w{\n+1^*}{\c}{\k-1/2} - \density{\n}{\c}{\k+1/2} \w{\n+1^*}{\c}{\k+1/2}\right)
\end{equation}

%=======================================================================================================================
\clearpage
\subsection{Virtual potential temperature}
\label{sub:virtual_potential_temperature}
%=======================================================================================================================
\begin{equation}
  \vpotempprime{\n+1^*}{\c}{\k} = \frac{\density{\n}{\c}{\k}\vpotempprime{\n}{\c}{\k}}{\density{\n+1^*}{\c}{\k}} \left( \left( \frac{\exnerprime{\n+1^*}{\c}{\k}}{\exnerprime{\n}{\c}{\k}} -1 \right) \frac{\cvd}{\Rd} +1 \right)
\end{equation}

%=======================================================================================================================
\clearpage
\subsection{Miura scheme}
\label{sub:miura}
%=======================================================================================================================

\begin{align}
  &\density{\n}{\e}{\k} = \densityref{\e}{\k} + \densityprime{\n}{\c 0/1}{\k} + \Delta^{\hat{x}}_{\c - \text{btrj}} \densityprimedx{\n}{\c}{\k} + \Delta^{\hat{y}}_{\c - \text{btrj}} \densityprimedy{\n}{\c}{\k} \\
  &\vpotemp{\n}{\e}{\k} = \vpotempref{\e}{\k} + \vpotempprime{\n}{\c 0/1}{\k} + \Delta^{\hat{x}}_{\c - \text{btrj}} \vpotempprimedx{\n}{\c}{\k} + \Delta^{\hat{y}}_{\c - \text{btrj}} \vpotempprimedy{\n}{\c}{\k}
\end{align}

\begin{align}
  & \densityprimedx{\n}{\c}{\k} = \sum_{\offProv{c2e2co}} \geofacgrgx\ \densityprime{\n}{\c}{\k} \\
  & \densityprimedy{\n}{\c}{\k} = \sum_{\offProv{c2e2co}} \geofacgrgy\ \densityprime{\n}{\c}{\k} \\
  & \vpotempprimedx{\n}{\c}{\k} = \sum_{\offProv{c2e2co}} \geofacgrgx\ \vpotempprime{\n}{\c}{\k} \\
  & \vpotempprimedy{\n}{\c}{\k} = \sum_{\offProv{c2e2co}} \geofacgrgy\ \vpotempprime{\n}{\c}{\k}
\end{align}

%=======================================================================================================================
\clearpage
\subsection{Diffusion}
\label{sub:diffusion}
%=======================================================================================================================

%-------------------------------------------------------------------------------
\paragraph{e2v interpolation}

\begin{align}
    & \windu{}{\v}{} = \sum_{\offProv{v2e}} \Wrbf_1 \vn{}{\e}{} \\
    & \windv{}{\v}{} = \sum_{\offProv{v2e}} \Wrbf_2 \vn{}{\e}{}
\end{align}


%-------------------------------------------------------------------------------
\paragraph{Smagorinsky coefficient}

% Intermediate steps
% \begin{align}
%   & \dvttang = - \left( \windu{}{\v 0}{} \orientationtangx{\v 0} + \windv{}{\v 0}{} \orientationtangy{\v 0} \right)
%                + \left( \windu{}{\v 1}{} \orientationtangx{\v 1} + \windv{}{\v 1}{} \orientationtangy{\v 1} \right) \\
%   & \dvtnorm = - \left( \windu{}{\v 2}{} \orientationtangx{\v 2} + \windv{}{\v 2}{} \orientationtangy{\v 2} \right)
%                + \left( \windu{}{\v 3}{} \orientationtangx{\v 3} + \windv{}{\v 3}{} \orientationtangy{\v 3} \right)
% \end{align}
%
% \begin{align}
%   & \khsmagone = - \left( \windu{}{\v 0}{} \orientationnormx{\v 0} + \windv{}{\v 0}{} \orientationnormy{\v 0} \right)
%                  + \left( \windu{}{\v 1}{} \orientationnormx{\v 1} + \windv{}{\v 1}{} \orientationnormy{\v 1} \right) \\
%   & \khsmagone = \frac{\khsmagone\ \orientationtang{\e}}{\Ledge} + \frac{\dvtnorm}{\Lvertvert} \\
%   %
%   & \khsmagtwo = - \left( \windu{}{\v 2}{} \orientationnormx{\v 2} + \windv{}{\v 2}{} \orientationnormy{\v 2} \right)
%                  + \left( \windu{}{\v 3}{} \orientationnormx{\v 3} + \windv{}{\v 3}{} \orientationnormy{\v 3} \right) \\
%   & \khsmagtwo = \frac{\khsmagtwo}{\Lvertvert} - \frac{\dvttang\ \orientationtang{\e}}{\Ledge} \\
%   %
%   & \Khsmag{}{ec}{} = \Khsmag{}{\e}{} = \diffmultfacsmag \sqrt{ \khsmagtwo^2 + \khsmagone^2 } \\
%   & \Khsmag{\n}{\e}{\k} = \min{\left(\smaglimit, \max{\left(0, \Khsmag{}{\e}{} - \smagoffset \right)} \right)}
% \end{align}

\begin{equation}
  \begin{aligned}
    & \Khsmag{\n}{\e}{\k} &&= \min{\left(\smaglimit, \max{\left(0, \diffmultfacsmag \sqrt{ * } - \smagoffset \right)} \right)} \\
    & * &&= \left(
            \frac{
                - \left( \windu{}{\v 2}{} \orientationnormx{\v 2} + \windv{}{\v 2}{} \orientationnormy{\v 2} \right)
                + \left( \windu{}{\v 3}{} \orientationnormx{\v 3} + \windv{}{\v 3}{} \orientationnormy{\v 3} \right)
            }{\Lvertvert}
            - \frac{\left(
                - \left( \windu{}{\v 0}{} \orientationtangx{\v 0} + \windv{}{\v 0}{} \orientationtangy{\v 0} \right)
                + \left( \windu{}{\v 1}{} \orientationtangx{\v 1} + \windv{}{\v 1}{} \orientationtangy{\v 1} \right)
            \right)\orientationtang{\e}}{\Ledge}
        \right)^2 \\
    &   &&+ \left(
            \frac{\left(
                - \left( \windu{}{\v 0}{} \orientationnormx{\v 0} + \windv{}{\v 0}{} \orientationnormy{\v 0} \right)
                + \left( \windu{}{\v 1}{} \orientationnormx{\v 1} + \windv{}{\v 1}{} \orientationnormy{\v 1} \right)
            \right)\orientationtang{\e}}{\Ledge}
            + \frac{
                - \left( \windu{}{\v 2}{} \orientationtangx{\v 2} + \windv{}{\v 2}{} \orientationtangy{\v 2} \right)
                + \left( \windu{}{\v 3}{} \orientationtangx{\v 3} + \windv{}{\v 3}{} \orientationtangy{\v 3} \right)
            }{\Lvertvert}
        \right)^2
  \end{aligned}
\end{equation}

%-------------------------------------------------------------------------------
\paragraph{Nabla 2}

\begin{equation}
  \vnlapl{\n}{\e}{\k} = 4 \left[
      \frac{
          \windu{}{\v 0}{} \orientationnormx{\v 0} + \windv{}{\v 0}{} \orientationnormy{\v 0}
        + \windu{}{\v 1}{} \orientationnormx{\v 1} + \windv{}{\v 1}{} \orientationnormy{\v 1}
        - 2 \vn{}{\e}{}
      }{\Ledge^2}
      + \frac{
          \windu{}{\v 2}{} \orientationnormx{\v 2} + \windv{}{\v 2}{} \orientationnormy{\v 2}
        + \windu{}{\v 3}{} \orientationnormx{\v 3} + \windv{}{\v 3}{} \orientationnormy{\v 3}
        - 2 \vn{}{\e}{}
      }{\Lvertvert^2}
    \right]
\end{equation}

\begin{equation}
  \wlapl{}{\c}{\k-1/2} = \sum_{\offProv{c2e2co}} \Whor \w{}{\c}{\k-1/2}
\end{equation}

\begin{align}
    %& \temparray{\n}{\e}{\k} = \Khsmag{\n}{\e}{\k} \Cgrad \Gradn_{\offProv{e2c}} \vpotempprime{\n}{\c}{\k} \\
    %& \vpotempprimelapl{\n}{\c}{\k} = \sum_{\offProv{c2e}} \Whor \temparray{\n}{\e}{\k}\\
    & \vpotempprimelapl{\n}{\c}{\k} = \sum_{\offProv{c2e}} \left( \Whor \Khsmag{}{\e}{\k} \Cgrad \Gradn_{\offProv{e2c}} \vpotempprime{\n}{\c}{\k} \right)
\end{align}


%-------------------------------------------------------------------------------
\paragraph{e2v interpolation}

\begin{align}
  & \vnlapl{}{\v^x}{} = \sum_{\offProv{v2e}} \Wrbf_1 \vnlapl{}{\e}{} \\
  & \vnlapl{}{\v^y}{} = \sum_{\offProv{v2e}} \Wrbf_2 \vnlapl{}{\e}{}
\end{align}

%-------------------------------------------------------------------------------
\paragraph{Nabla 4}

\begin{align}
  & \nabvtang = \vnlapl{}{\v^x 0}{} \orientationnormx{\v 0} + \vnlapl{}{\v^y 0}{} \orientationnormy{\v 0}
              + \vnlapl{}{\v^x 1}{} \orientationnormx{\v 1} + \vnlapl{}{\v^y 1}{} \orientationnormy{\v 1}
  \\
  & \nabvnorm = \vnlapl{}{\v^x 2}{} \orientationnormx{\v 2} + \vnlapl{}{\v^y 2}{} \orientationnormy{\v 2}
              + \vnlapl{}{\v^x 3}{} \orientationnormx{\v 3} + \vnlapl{}{\v^y 3}{} \orientationnormy{\v 3}
  \\
  & \vnlapllapl{\n}{\e}{\k} = 4 \left[
            \frac{ \nabvnorm - 2 \vnlapl{\n}{\e}{\k} }{\Lvertvert^2}
          + \frac{ \nabvtang - 2 \vnlapl{\n}{\e}{\k} }{\Ledge^2}
      \right]
\end{align}


%-------------------------------------------------------------------------------
\paragraph{Final updates}

\begin{align}
  & \vn{d}{\e}{\k} = \vn{\n}{\e}{\k} + \edgearea \left( \Khsmag{}{\e}{\k} \vnlapl{\n}{\e}{\k} - \diffmultfacvn \edgearea \vnlapllapl{\n}{\e}{\k} \right) \\
  & \w{d}{\c}{\k-1/2} = \w{\n}{\c}{\k-1/2} - \diffmultfacw \cellarea^2 \sum_{\offProv{c2e2co}} \Whor \wlapl{}{\c}{\k-1/2} \\
  & \vpotemp{d}{\c}{\k} = \vpotemp{\n}{\c}{\k} + \cellarea \vpotempprimelapl{\n}{\c}{\k} \\
  & \exner{d}{\c}{\k} = \exner{\n}{\c}{\k} \left( 1 + \frac{\Rd}{\cvd} \left( \frac{\vpotemp{d}{\c}{\k}}{\vpotemp{\n}{\c}{\k}} -1 \right) \right)
\end{align}

%=======================================================================================================================
\clearpage
\subsection{Don't remember what these are for}
\label{sub:misc}
%=======================================================================================================================

\begin{align}
    &\exnerprimedzz{\ntilde}{\c}{\k} = - \frac{1}{2} \left( (\vpotempprime{\n}{\c}{\k-1/2} - \vpotempprime{\n}{\c}{\k+1/2}) \dexrefdz{\c}{\k} + \vpotempprime{\n}{\c}{\k} \ddexrefdzz{\c}{\k} \right), \quad \k \in [\nflatgradp, \nlev) \\
    &\ddz{\exnerref{}{}} = - \frac{g}{\cpd \vpotempref{}{}}
\end{align}



%%%%%%%%%%%%%%%%%%%%%%%%%%%%%%%%%%%%%%%%%%%%%%%%%%%%%%%%%%%%%%%%%%%%%%%%%%%%%%%%%%%%%%%%%%%%%%%%%%%%%%%%%%%%%%%%%%%%%%%%
\end{document}
