% ICON4Py - ICON inspired code in Python and GT4Py
%
% Copyright (c) 2022-2024, ETH Zurich and MeteoSwiss
% All rights reserved.
%
% Please, refer to the LICENSE file in the root directory.
% SPDX-License-Identifier: BSD-3-Clause

%===============================================================================
% This file contains macro definitions that can be used in docstrings and are
% parsed by the MathJax processor within sphinx.
% These macros are loaded into `mathjax3_config` by conf.py
% *** AT THE MOMENT ONLY SINGLE LINE MACROS ARE WORKING ***

% The predefined color names are
% black, blue, brown, cyan, darkgray, gray, green, lightgray, lime, magenta,
% olive, orange, pink, purple, red, teal, violet, white, yellow.

%===============================================================================
% Maths

% Glyph sizes
\newcommand{\superscriptsize}{\scriptsize}
\newcommand{\subscriptsize}{\scriptsize}

% Differentials and deltas
\newcommand{\dif}{\mathrm{d}}
\newcommand{\Dt}{{\Delta t}}
\newcommand{\Dz}{{\Delta z}}

% Derivatives
\newcommand{\ddt}[1]{\frac{\dif #1}{\dif t}}
\newcommand{\ddz}[1]{\frac{\dif #1}{\dif z}}
\newcommand{\Padt}[1]{\frac{\partial #1}{\partial t}}
\newcommand{\Padz}[1]{\frac{\partial #1}{\partial z}}

% Variables
\newcommand{\vn}[2]{{v_n}^{\superscriptsize #1}_{\subscriptsize #2}}
\newcommand{\vt}[2]{{v_t}^{\superscriptsize #1}_{\subscriptsize #2}}
\newcommand{\vh}[2]{{v_h}^{\superscriptsize #1}_{\subscriptsize #2}}
\newcommand{\thetav}[2]{{\theta_v}^{\superscriptsize #1}_{\subscriptsize #2}}
\newcommand{\thetavprime}[2]{{\theta_v'}^{\superscriptsize #1}_{\subscriptsize #2}}
\newcommand{\exner}[2]{{\pi}^{\superscriptsize #1}_{\subscriptsize #2}}
\newcommand{\exnerprime}[3]{{\pi'}^{\superscriptsize\color{teal}#1}_{\subscriptsize{\color{blue}#2},\ {\color{orange}#3}}}

% Physical constants
\newcommand{\cpd}{c_{pd}}
\newcommand{\cvd}{c_{vd}}

% Numerical constants
\newcommand{\nlev}{{\texttt{num_lev}}}

% Temporal steps
\newcommand{\n}{{\color{teal}n}}
\newcommand{\ntilde}{{\color{teal}\tilde{\n}}}

# Vertical levels
\renewcommand{\k}{{\color{orange}k}}

% Horizontal discretization
\newcommand{\c}{{\color{blue}c}}
\newcommand{\e}{{\color{blue}e}}
\newcommand{\v}{{\color{blue}v}}