% ICON4Py - ICON inspired code in Python and GT4Py
%
% Copyright (c) 2022-2024, ETH Zurich and MeteoSwiss
% All rights reserved.
%
% Please, refer to the LICENSE file in the root directory.
% SPDX-License-Identifier: BSD-3-Clause

%===============================================================================
% This file contains macro definitions that can be used in docstrings and are
% parsed by the MathJax processor within Sphinx.
% These macros are loaded into `mathjax3_config` in conf.py

% The predefined color names are defined on this page
% https://en.wikibooks.org/wiki/LaTeX/Colors
% both the standard as well as the dvipsnames can be used

%===============================================================================
% Aesthetic macros

\renewcommand{\sum}{\Large{\Sigma}} % move sub/superscripts of sum symbol

% Glyph sizes
\newcommand{\superscriptsize}{\scriptscriptstyle}
\newcommand{\subscriptsize}{\scriptscriptstyle}

% Colors
\newcommand{\timecolor}[1]{\color{teal}{#1}}
\newcommand{\vertcolor}[1]{\color{YellowOrange}{#1}}
\newcommand{\horicolor}[1]{\color{RedViolet}{#1}}

%===============================================================================
% Numerics

% Discretization points
\newcommand{\timestep}[1]{{\subscriptsize{\timecolor{#1}}}}
\newcommand{\vertlevel}[1]{{\subscriptsize{\vertcolor{#1}}}}
\newcommand{\horipoint}[1]{{\subscriptsize{\horicolor{#1}}}}

% Differentials and deltas
\newcommand{\dif}{\mathrm{d}}
\newcommand{\Dt}{{\Delta t}}
\newcommand{\Dz}[1]{{\Delta z}_\vertlevel{#1}}
\newcommand{\divhori}[1]{\vec{\nabla}\!_h\cdot\left(#1\right)}
\newcommand{\Gradn}{{{\Large{\Delta}}}}

% Derivatives
\newcommand{\ddt}[1]{\frac{\dif #1}{\dif t}}
\newcommand{\ddz}[1]{\frac{\dif #1}{\dif z}}
\newcommand{\ddxn}[1]{\frac{\dif #1}{\dif n}}
\newcommand{\ddxt}[1]{\frac{\dif #1}{\dif t}}
\newcommand{\pdt}[1]{\frac{\partial #1}{\partial t}}
\newcommand{\pdz}[1]{\frac{\partial #1}{\partial z}}
\newcommand{\pdzz}[1]{\frac{\partial^2 #1}{\partial z^2}}
\newcommand{\pdxn}[1]{\frac{\partial #1}{\partial n}}
\newcommand{\pdxt}[1]{\frac{\partial #1}{\partial t}}

% Variables with helper macro for formatting superscript and subscript
\newcommand{\variable}[4]{{{#1}_{\horipoint{#3}\,\vertlevel{#4}}^{\timestep{#2}}}}
\newcommand{\vn}[3]{          \variable{v_n            }{#1}{#2}{#3}}
\newcommand{\vt}[3]{          \variable{v_t            }{#1}{#2}{#3}}
\newcommand{\vh}[3]{          \variable{\vec{v_h}      }{#1}{#2}{#3}}
\newcommand{\w}[3]{           \variable{w              }{#1}{#2}{#3}}
\newcommand{\wcc}[3]{         \variable{w_{\text{cc}}  }{#1}{#2}{#3}}
\newcommand{\kinehori}[3]{    \variable{K_h            }{#1}{#2}{#3}}
\newcommand{\advvn}[3]{       \variable{\text{adv}(v_n)}{#1}{#2}{#3}}
\newcommand{\vortvert}[3]{    \variable{\zeta          }{#1}{#2}{#3}}
\newcommand{\density}[3]{     \variable{\rho           }{#1}{#2}{#3}}
\newcommand{\pres}[3]{        \variable{P              }{#1}{#2}{#3}}
\newcommand{\vtemp}[3]{       \variable{T_v            }{#1}{#2}{#3}}
\newcommand{\vtempprime}[3]{  \variable{T_v'           }{#1}{#2}{#3}}
\newcommand{\vpotemp}[3]{     \variable{\theta_v       }{#1}{#2}{#3}}
\newcommand{\vpotempprime}[3]{\variable{\theta_v'      }{#1}{#2}{#3}}
\newcommand{\exner}[3]{       \variable{\pi            }{#1}{#2}{#3}}
\newcommand{\exnerprime}[3]{  \variable{\pi'           }{#1}{#2}{#3}}
\newcommand{\exnergradh}[3]{  \variable{\pdxn{\exnerprime{#1}{}{}}}{}{#2}{#3}}
%
\newcommand{\variableX}[2]{{{#1}_{\horipoint{#2}}}}
\newcommand{\coriolis}[1]{    \variableX{f                                                                 }{#1}}
\newcommand{\exnhydrocorr}[1]{\variableX{\frac{g}{\cpd \vpotemp{\color{black}{2}}{}{}}\pdxn{\vpotemp{}{}{}}}{#1}}
%
\newcommand{\variableXK}[3]{{{#1}_{\horipoint{#2}\,\vertlevel{#3}}}}
\newcommand{\presref}[2]{   \variableXK{P_0            }{#1}{#2}}
\newcommand{\vpotempref}[2]{\variableXK{\theta_{v0}    }{#1}{#2}}
\newcommand{\exnerref}[2]{  \variableXK{\pi_0          }{#1}{#2}}

% Physical constants
\newcommand{\cpd}{c_{pd}}
\newcommand{\cvd}{c_{vd}}

% Numerical constants
\newcommand{\nlev}{{\texttt{num_lev}}}
\newcommand{\nflatlev}{{\texttt{flat_lev}}}
\newcommand{\nflatgradp}{{\texttt{flat_gradp_lev}}}

% Temporal steps
\newcommand{\n}{{\timecolor{n}}}
\newcommand{\ntilde}{{\timecolor{\tilde{\n}}}}

% Vertical levels
\renewcommand{\k}{{\vertcolor{k}}}

% Horizontal discretization
\renewcommand{\c}{{\horicolor{c}}}
\newcommand{\e}{{\horicolor{e}}}
\renewcommand{\v}{{\horicolor{v}}}

% Horizontal offsets
\newcommand{\offProv}[1]{\horicolor{#1}}

% Inter/extrapolation and derivative weights
\newcommand{\Wrbf}{\overset{\scriptscriptstyle{\text{rbf}}}{\chi}}
\newcommand{\Wlev}{\overset{\scriptscriptstyle{\text{lev}}}{\chi}}
\newcommand{\Whor}{\overset{\scriptscriptstyle{\text{hor}}}{\chi}}
\newcommand{\Cgrad}{\overset{\scriptscriptstyle{\text{grad}}}{\chi}}
\newcommand{\Crot}{\overset{\scriptscriptstyle{\text{rot}}}{\chi}}
\newcommand{\WtimeExner}{\gamma}
\newcommand{\WtimeVWind}{\eta}

% Indexes
\newcommand{\IDXpg}{\overset{\scriptstyle{\partial \pi / \partial n}}{\text{IDX}}}
